\documentclass{article}
\usepackage[utf8]{inputenc}


\usepackage{amsmath}
\usepackage{amssymb} 
\usepackage{amsthm}  
\usepackage{dsfont}
\usepackage{mathrsfs}
\usepackage{mathtools}

\usepackage{geometry}

\usepackage{hyperref}        


\usepackage[french]{babel}

\usepackage[shortlabels]{enumitem}

\newcommand{\indep}{\perp\!\!\! \perp}

\theoremstyle{definition} 
\newtheorem{definition}{Définition}

\theoremstyle{definition} 
\newtheorem{prop}{Proposition}

\theoremstyle{definition}
\newtheorem{coro}{Corollaire}

\theoremstyle{plain}
\newtheorem{example}{Exemple}

\theoremstyle{theorem}
\newtheorem{theorem}{Théorème}

\theoremstyle{theorem}
\newtheorem{lemma}{Lemme}

\begin{document}

\title{Logique Notes}
\author{Yago iglesias}
\maketitle
\tableofcontents


\section{Ordres}

\subsection{Les axiomes des ordres}


\begin{definition}[Relation d'ordre]

    On dit que la relation $\leqslant$ sur un ensemble $E$ non vide est une relation d'ordre au sens large sur $E$ si elle est :
    \begin{itemize}
        \item Réflexive : $\forall x \in E$, $x \leqslant x$
        \item Anti-symétrique : $\forall x$, $y \in E$, ($x \leqslant y$ et $y \leqslant x$) $\implies x = y$)
        \item Transitive : $\forall x$, $y$, $z \in E$, ($x \leqslant y$ et $y \leqslant z$) $\implies x \leqslant z$
    \end{itemize}
    
    Aussi, on dit que la relation $<$ sur un ensemble $E$ non vide est une relation d'ordre au sens strict sur $E$ si :
    \begin{itemize}
        \item $\forall x$, $y \in E$, $x < y$ si et seulement si $x \leqslant y$ et $x \ne y$
    \end{itemize}
    On peut l'axiomatiser de la manière suivante :

    $<$ est une relation d'ordre strict sur $E$ si elle est :
    \begin{itemize}
        \item Irréflexive : $x \nless x$
        \item Transitive : $\forall x$, $y$, $z \in E$, ($x < y$ et $y < z$) $\implies x < z$
    \end{itemize}
    On peut aussi maintenant définir $\leqslant$ de la manière suivante :
    \begin{itemize}
        \item $x \leq y$ si et seulement si  $x < y$ ou $x = y$
    \end{itemize}

\end{definition}
    

\begin{definition}[Ensemble ordonné]

    On dit qu'un ensemble ($E$, $\leqslant$) est ordonné si $E$ est non vide et qu'il est muni d'une relation d'ordre $\leqslant$.

\end{definition}

\begin{definition}[Relation totale]

    On dit qu'une relation d'ordre $\leqslant$ est totale sur $E$ si pour $\forall x$, $y \in E$, x et y sont comparables.

\end{definition}

\begin{lemma}

    Soit $<$ une relation d'ordre strict, soit $\leqslant$ l'ordre large associé. Donc $\leqslant$ est totale si et seulement si :
	\begin{equation*}
        \forall x \text{, } y \in E (x < y \text{ ou } y < x \text{ ou } x = y \text{ (Trichotomie)})
	\end{equation*}
	\noindent

\end{lemma}

\begin{example}
    \begin{itemize}
        \item $x \sqsubseteq y$ si et seulement si $|x| \leqslant |y|$ pour $x$, $y \in \mathbb{C}$, n'est pas anti-symétrique
            car $-1 \sqsubseteq 1 \sqsubseteq -1$: ce n'est pas une relation d'ordre large.
        \item ($\mathscr{P}(\mathbb{N})$, $\subseteq$): Ensemble ordonné d'ordre large pas total (${2} \nsubseteq {17}$ et ${17} \nsubseteq {2}$)
        \item ($\mathbb{N}$, $\leqslant$): Ensemble ordonné d'ordre large total, et qui admet en plus un minimum
        \item ($\mathbb{Q}$, $\leqslant$): Ensemble ordonné d'ordre large total, et ensemble qui est dense
        \item ($\mathbb{R}$, $\leqslant$): Ensemble ordonné d'ordre large total, et ensemble qui est dense
    \end{itemize} 
\end{example}

\subsection{Morphismes d'ordre}

\begin{definition}

    Un morphisme d'ordre entre 2 ensembles d'ordre ($A$, $\leqslant_A$), ($B$, $\leqslant_B$) est une application $\varphi : A \to B$
    tel que $\forall x$, $y\in A$, $x \leqslant_A y \iff \varphi(x) \leqslant_B \varphi(y)$

\end{definition}

\begin{definition}

    Une relation d'ordre $\leqslant$ sur $E$ définit un bon ordre sur $E$ si :
    \begin{itemize}
        \item $\leqslant$ est un ordre total
        \item Tous sous-ensemble de $E$ non vide a un plus petit élément, c'est-à-dire pour tout $A \subseteq E$, $A \neq \emptyset$, il existe
            $a \in A$ tel que $\forall b \in A$, $a \leqslant b$
    \end{itemize}

\end{definition}

\begin{example}
    \begin{itemize}
        \item ($\mathbb{N}$, $\leqslant$): Bon ordre
        \item ($\mathbb{Z}$, $\leqslant$): Mauvais ordre
        \item ($[0, 1]$, $\leqslant$): Mauvais ordre ($]0, 1]$ n'admet pas de minimum)
        \item ($\mathscr{P}(\mathbb{N})$, $\subseteq$): Mauvais ordre : Pas total
        \item ($\emptyset$, $\emptyset$) : Bon ordre
    \end{itemize} 
\end{example}

\begin{prop}
    Soit ($A$, $\leqslant_A$) un bon ordre. Soit $B \subseteq A$ et soit $\leqslant_B$ une relation d'ordre large de B, alors ($B$, $\leqslant_B$) est bien ordonnée.
\end{prop}

\begin{proof}

    Soit $C \subseteq B$ non vide, alors $C \subseteq A$ non vide. Donc il existe $c \in C$ l'élément le plus petit de $C$ par rapport à $\leqslant_A$. On
    veut que $c$ soit l'élément le plut petit pour $\leqslant_B$. Soit $d \in C$, donc $c \leqslant_A d$ et puis $c \leqslant_B d$.

\end{proof}

\begin{example}
    \begin{itemize}
        \item (${\frac{1}{n} | n \in \mathbb{N}^*}$, $\leqslant$) n'est pas un bon ordre
        \item (${-\frac{1}{n} | n \in \mathbb{N}^*}$, $\leqslant$) est un bon ordre : il existe un isomorphisme (${-\frac{1}{n} | n \in \mathbb{N}^*}$, $\leqslant$) $\to$ ($\mathbb{N}$, $\leqslant$)
    \end{itemize}
\end{example}

\begin{prop}

    Un ensemble non vide total ordonné ($E$, $\leqslant$) est bien ordonné si et seulement si il vérifie la propriété de récurrence bien fondée pour tout sous ensemble $J \subseteq E$ :
    \begin{equation*}
        \forall y \in E \text{, } \forall z \in E \text{, avec } z < y \text{, alors } (z \in J \implies y \in J) \implies J = E
    \end{equation*}
	\noindent

\end{prop}

\begin{proof}

    On reformule par contraposée la proposition de récurrence bien fondée :
    \begin{equation*}
        \exists x \in E\backslash J \implies \exists y \in E\backslash J \text{ tel que } \forall z \in E \text{, } z < y \implies z \in J \text{ avec } y = min(E\backslash J)
    \end{equation*}
	\noindent
    C'est la proposition des bons ordres pour $E\backslash J \subseteq E$.

\end{proof}

\begin{example}
    Pour $x$, $y \in \mathbb{Z}$, on écrit $x | y$ si et seulement si $\exists z \in \mathbb{Z}$ tel que $y = x\cdot z$
    \begin{itemize}
        \item ($\mathbb{Z}$, $|$) n'est pas un bon ordre car $-1 | 1 | -1$ et $-1 \neq 1$
        \item ($\mathbb{N}$, $|$) est une relation d'ordre pas totale car $2 \nmid 3 \nmid 2$
        \item ($\mathbb{N}$, $\leqslant$) bon ordre car il n'existe pas de suite infinie décroissante dans $\mathbb{N}$
    \end{itemize}
\end{example}

\end{document}
