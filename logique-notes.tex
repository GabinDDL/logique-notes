\documentclass{article}
\usepackage[utf8]{inputenc}
\usepackage[T1]{fontenc}


\usepackage{amsmath}
\usepackage{amssymb} 
\usepackage{amsthm}  
\usepackage{dsfont}
\usepackage{mathrsfs}
\usepackage{mathtools}

\usepackage{geometry}

\usepackage{hyperref}        


\usepackage[french]{babel}

\usepackage[shortlabels]{enumitem}


\usepackage{fancyhdr}

\fancypagestyle{toc}{%
\fancyhf{}%
\fancyhead[L]{\rightmark}%
\fancyhead[R]{\thepage}%
}

\pagestyle{toc}

\newcommand{\indep}{\perp\!\!\! \perp}

\theoremstyle{plain}
\newtheorem{theorem}{Théorème}[section]
\newtheorem{coro}[theorem]{Corollaire}
\newtheorem{lemma}{Lemme}[section]
\newtheorem{prop}{Proposition}[section]

\theoremstyle{definition} 
\newtheorem{definition}{Définition}[section]
\newtheorem{example}{Exemple}[subsection]
\newtheorem{exercice}{Exercice}[subsection]

\theoremstyle{plain}
\newtheorem{remarque}{Remarque}[subsection]


\begin{document}
\begin{titlepage}
	\newcommand{\HRule}{\rule{\linewidth}{0.5mm}}
	\center

	\HRule\\[0.4cm]

	\textsc{\Large Logique}\\[0.5cm]
	\textsc{\large Un ensemble compréhensible de notes de cours}\\[0.5cm]

	\HRule\\[1.5cm]


	{\large\textit{Auteur}}\\
	Yago \textsc{Iglesias}


	\vfill\vfill\vfill

	{\large\today}

	\vfill

\end{titlepage}

\tableofcontents

\section{Introduction}

Ce document est un recueil de notes de cours sur la logique niveau L3. Il est
basé sur les cours de Mme.~\textsc{Sylvy Anscombe} à Université Paris Cité, cependant toute
erreur ou inexactitude est de ma responsabilité.
Si bien \textsc{Yago IGLESIAS} est l'auteur de ce document, il n'est pas
le seul contributeur. Un remerciement particulier à \textsc{Gabin Dudillieu} pour sa
participation active à la rédaction de ce document. Tout futur contributeur
peut se retouver dans dans la section contributeurs du répertoire
\href{https://github.com/Yag000/logique-notes/graphs/contributors}{GitHub}.
\vspace{0.5cm}

Toute erreur ou remarque est la bienvenue.
Sentez vous libres de contribuer à ce document par le biais de \href{https://github.com/Yag000/logique-notes}{GitHub},
où vous pouvez trouver le code source de ce document et une version pdf à jour.
Si vous n'etes pas familiers avec \textit{Git} ou \LaTeX , vous pouvez toujours me contacter
par \href{mailto: yago.iglesias.vazquez@gmail.com}{mail}.




\section{Ordres}

\subsection{Les axiomes des ordres}


\begin{definition}[Relation d'ordre]

	On dit que la relation $\leqslant$ sur un ensemble $E$ non vide est une relation d'ordre au sens large sur $E$ si elle est :
	\begin{itemize}
		\item Réflexive : $\forall x \in E$, $x \leqslant x$
		\item Anti-symétrique : $\forall x$, $y \in E$, ($x \leqslant y$ et $y \leqslant x$) $\implies x = y$)
		\item Transitive : $\forall x$, $y$, $z \in E$, ($x \leqslant y$ et $y \leqslant z$) $\implies x \leqslant z$
	\end{itemize}

	Aussi, on dit que la relation $<$ sur un ensemble $E$ non vide est une relation d'ordre au sens strict sur $E$ si :
	\begin{equation*}
		\forall x,\,y \in E,\,x < y \iff x \leqslant y \quad \text{et} \quad x \ne y
	\end{equation*}
	On peut l'axiomatiser de la manière suivante :

	$<$ est une relation d'ordre strict sur $E$ si elle est :
	\begin{itemize}
		\item Irréflexive : $x \nless x$
		\item Transitive : $\forall x$, $y$, $z \in E$, ($x < y$ et $y < z$) $\implies x < z$
	\end{itemize}
	On peut aussi maintenant définir $\leqslant$ de la manière suivante :
	\begin{equation*}
		x \leq y \iff  x < y \quad \text{ou} \quad  x = y
	\end{equation*}

\end{definition}


\begin{definition}[Ensemble ordonné]

	On dit qu'un ensemble ($E$, $\leqslant$) est ordonné si $E$ est non vide et qu'il est muni d'une relation d'ordre $\leqslant$.

\end{definition}

\begin{definition}[Relation totale]

	On dit qu'une relation d'ordre $\leqslant$ est totale sur $E$ si pour $\forall x$, $y \in E$, x et y sont comparables.

\end{definition}

\begin{lemma}

	Soit $<$ une relation d'ordre strict, soit $\leqslant$ l'ordre large associé, $\leqslant$ est totale si et seulement si :
	\begin{equation*}
		\forall x \text{, } y \in E (x < y \text{ ou } y < x \text{ ou } x = y \text{ (Trichotomie)})
	\end{equation*}

\end{lemma}

\begin{example}
	\begin{itemize}
		\item $x \sqsubseteq y$ si et seulement si $|x| \leqslant |y|$ pour $x$, $y \in \mathbb{C}$, n'est pas anti-symétrique
		      car $-1 \sqsubseteq 1 \sqsubseteq -1$: ce n'est pas une relation d'ordre large car $1 \neq -1$.
		\item ($\mathscr{P}(\mathbb{N})$, $\subseteq$): Ensemble ordonné d'ordre large pas total ($\{2\} \nsubseteq \{17\}$ et $\{17\} \nsubseteq \{2\}$)
		\item ($\mathbb{N}$, $\leqslant$): Ensemble ordonné d'ordre large total, et qui admet en plus un minimum
		\item ($\mathbb{Q}$, $\leqslant$): Ensemble ordonné d'ordre large total, et ensemble qui est dense
		\item ($\mathbb{R}$, $\leqslant$): Ensemble ordonné d'ordre large total, et ensemble qui est dense
	\end{itemize}
\end{example}

\subsection{Morphismes d'ordres}

\begin{definition}

	Un morphisme d'ordres entre 2 ensembles d'ordre ($A$, $\leqslant_A$), ($B$, $\leqslant_B$) est une application $\varphi : A \to B$
	tel que $\forall x$, $y\in A$, $x \leqslant_A y \iff \varphi(x) \leqslant_B \varphi(y)$

\end{definition}

\begin{definition}
	Un isomorphisme d'ordres est un morphisme d'ordres bijectif.
\end{definition}

\begin{remarque}
	Un isomorphisme est une bijection croissante dont la réciproque est croissante.
\end{remarque}

\begin{definition}
	Une relation d'ordre $\leqslant$ sur $E$ définit un bon ordre sur $E$ si:
	\begin{itemize}
		\item $\leqslant$ est un ordre total
		\item Tous sous-ensemble de $E$ non vide a un plus petit élément, c'est-à-dire pour tout $A \subseteq E$, $A \neq \emptyset$, il existe
		      $a \in A$ tel que $\forall b \in A$, $a \leqslant b$
	\end{itemize}
\end{definition}

\begin{example}
	\begin{itemize}
		\item ($\mathbb{N}$, $\leqslant$): Bon ordre
		\item ($\mathbb{Z}$, $\leqslant$): N'est pas un bon ordre car $]-\infty, 0]$ n'est pas borné à gauche
		\item ($[0, 1]$, $\leqslant$): N'est pas un bon ordre ($]0, 1]$ n'admet pas de minimum)
		\item ($\mathscr{P}(\mathbb{N})$, $\subseteq$): N'est pas un bon ordre: Pas total
		\item ($\emptyset$, $\emptyset$): Bon ordre
	\end{itemize}
\end{example}

\begin{prop}
	Soit ($A$, $\leqslant_A$) un bon ordre. Soit $B \subseteq A$ et soit $\leqslant_B$ une relation d'ordre large de B, alors ($B$, $\leqslant_B$) est bien ordonnée.
\end{prop}

\begin{proof}

	Soit $C \subseteq B$ non vide, alors $C \subseteq A$ non vide. Donc il existe $c \in C$ l'élément le plus petit de $C$ par rapport à $\leqslant_A$. On
	veut que $c$ soit l'élément le plut petit pour $\leqslant_B$. Soit $d \in C$, donc $c \leqslant_A d$ et puis $c \leqslant_B d$.

\end{proof}

\begin{example}
	\begin{itemize}
		\item ($\left\{\frac{1}{n} | n \in \mathbb{N}^*\right\}$, $\leqslant$) n'est pas un bon ordre
		\item ($\left\{-\frac{1}{n} | n \in \mathbb{N}^*\right\}$, $\leqslant$) est un bon ordre: il existe un isomorphisme ($\left\{-\frac{1}{n} | n \in \mathbb{N}^*\right\}$, $\leqslant$) $\to$ ($\mathbb{N}$, $\leqslant$)
	\end{itemize}
\end{example}

\begin{prop}

	Un ensemble non vide total ordonné ($E$, $\leqslant$) est bien ordonné si et seulement si il vérifie la propriété de récurrence bien fondée pour tout sous ensemble $J \subseteq E$:
	\begin{equation*}
		\forall y \in E \text{, } \forall z \in E \text{, avec } z < y \text{, alors } (z \in J \implies y \in J) \implies J = E
	\end{equation*}
	\noindent

\end{prop}

\begin{proof}

	On reformule par contraposée la proposition de récurrence bien fondée:
	\begin{equation*}
		\exists x \in E\setminus J \implies \exists y \in E\setminus J \text{ tel que } \forall z \in E \text{, } z < y \implies z \in J \text{ ce qui veut dire que } y = \min(E\setminus J)
	\end{equation*}
	\noindent
	C'est la proposition des bons ordres pour $E\setminus J \subseteq E$.

\end{proof}

\begin{example}
	Pour $x$, $y \in \mathbb{Z}$, on écrit $x | y$ si et seulement si $\exists z \in \mathbb{Z}$ tel que $y = x\cdot z$
	\begin{itemize}
		\item ($\mathbb{Z}$, $|$) n'est pas un bon ordre car $-1 | 1 | -1$ et $-1 \neq 1$
		\item ($\mathbb{N}$, $|$) est une relation d'ordre pas totale car $2 \nmid 3 \nmid 2$
		\item ($\mathbb{N}$, $\leqslant$) bon ordre car il n'existe pas de suite infinie décroissante dans $\mathbb{N}$
	\end{itemize}
\end{example}




\section{Axiomatisation de l'arithmétique}

\subsection{Introduction}

Nous avons besoin d'une définition explicite de $\mathbb{N}$.
Un exemple est la définition naïve suivante:

\begin{eqnarray*}
    0 &=& \emptyset \\
    1 &=& |\\
    2 &=& ||\\
    3 &=& ||\\
      &\cdots&
\end{eqnarray*}

où $S(x) = x \ |$. Mais cela est loin d'être pratique.

\subsection{Définition inductive}

On se place dans un univers avec deux symboles: $0$ et $S$.

De cette manière, l'ensemlbe des entiers naturels est défini comme le plus petit ensemble $\mathbb{N}$ qui contient
$0$ et qui est stable (clôt) par application du successeur $S$, i.e. si $x \in \mathbb{N}$ alors $S(x) \in \mathbb{N}$.

Notons $Cl(A)$ la clôture de $A$ par application de $S$ et le fait que $0 \in A$:
$$ Cl(A) = \{ 0 \in A \ \text{et} \ x \in \mathbb{N} \implies S(x) \in A \} $$

\begin{remarque}
    Si chacun des ensembles d'une famille $(A_i){i\in I}$ vérifie $Cl(A)$, alors leur intersection aussi.
\end{remarque}

Ainsi, $\mathbb{N} = \bigcap\limits_{Cl(A)} A$.

\subsection{Axiomatisation et les axiomes de Peano}

\begin{definition}[Axiomes pour les naturels]
	\begin{itemize}
		\item Succeseur non nul: $\forall x \in \mathbb{N}, S(x) \neq 0$
		\item Injectivité du successeur: $\forall x,y \in \mathbb{N}, S(x) = S(y) \implies x = y$
		\item Récurrence: Pour toute prorpiété $P$ << bien définie >> sur $\mathbb{N}$ alors
		      $$ \left( P(0) \wedge \forall x \in \mathbb{N}, P(x) \implies P(S(x))\right) \implies \forall x \in \mathbb{N}, P(x) $$
	\end{itemize}
\end{definition}

\begin{lemma}[Raissonement par récurrence]
	Tout entier est soit $0$ soit un successeur:
	$$ \forall x \in \mathbb{N}, x = 0 \vee \exists y \in \mathbb{N}, x = S(y) $$
\end{lemma}

\begin{proof}
	Par récurrence, la propriété $P(x) = x = 0 \vee \exists y \in \mathbb{N}, x = S(y)$.
	\begin{itemize}
		\item $P(0)$ est vraie car $0 = 0$.
		\item On suppose $P(x)$ vraie, donc $P(S(x))$.
	\end{itemize}
	Notons que dans cette preuve la propriété de récurrence n'est pas utilisée.
\end{proof}


\begin{definition}[Prédécesseur]
	Pour tout élément non nul, on appelle prédécesseur de $x$ un élément $y$ tel que $S(y) = x$.
\end{definition}

\begin{exercice}
	Montrer que les trois axiomes de Peano sont indépendants, c'est à dire, construire pour chacun des axiomes un
	contre-modèle $\mathscr{N}$, avec un élément distingué $0$ et une fonction $S : \mathscr{N} \to \mathscr{N}$, où l’axiome dont on
	veut montrer qu’il est indépendant n’est pas vérifié, mais les deux autres le sont.
\end{exercice}

\subsection{Définition par récurrence}

On peut définir $\leq$ par:
$$ x \leq y \iff \text{ il existe une suite finie } x_, S(x), S(S(x)), \ldots, S(S(\ldots S(x)) \ldots) = y $$
mais ceci n'est pas une définition du premier ordre.\\

L'objectif est de définir $+ : \mathbb{N} \times \mathbb{N} \to \mathbb{N}$:
\begin{itemize}
	\item $x + 0 = x$
	\item $x + S(y) = S(x + y)$
\end{itemize}

Mas pour cela nous avons besoin du théoreme suivant:

\begin{theorem} [Dedekind 1888]
	Si $E$ un ensemble non vide, $a \in E$ et $h : E \to E$ une fonction. Alors il existe une unique fonction $f : \mathbb{N} \to E$ telle que:
	\begin{itemize}
		\item $f(0) = a$
		\item $f(S(x)) = h(f(x))$
	\end{itemize}
\end{theorem}

\begin{proof}
	L'unicité se demontre par récurrence. Soit $g$ une fonction vérifiant les mêmes propriétés que $f$.
	\begin{itemize}
		\item $g(0) = a = f(0)$
		\item Si on a $g(x) = f(x)$ alors $f(S(x)) = h(f(x)) = h(g(x)) = g(S(x))$
	\end{itemize}
	Demontrons maintenant l'existence. Nous avons besoin d'une relation entre les éléments de $E$ et les éléments de $\mathbb{N}$, pour pouvoir
	définir une propriété de clôture. Pour cela on travaillera sur les sous ensembles de $\mathbb{N} \times E$.
	Cette propriété de clôture est inspirée par la définiton de $f$, pour les sous-ensembles $R$ de $\mathbb{N} \times E$.

	$$ Cl(A) = (0,a) \in R \ \text{et} \ \forall n \in \mathbb{N}, \forall y \in E, (n,y) \in R \implies (S(n), h(y)) \in R $$

	On peut vérifier que si $R$ est le graphe d'une fonction $f$ alors on retrouve les équations du théorème.

	L'ensemble des $R \subset \mathbb{N} \times E$ vérifiant la propriété de clôture est non vide car il contient $\mathbb{N} \times E$. On peut donc poser:
	$$ G = \bigcap\limits_{Cl(R)} R $$

	Il s'agit de montrer que $G$ est le graphe d'une fonction $f$ vérifiant les équations du théorème. Pour cela, il suffit de montrer que $G$ est une fonction.

	\begin{itemize}
		\item On a bien $Cl(G)$.
		\item Tout élément $n \in \mathbb{N}$ possède une unique image par la relation de graphe $G$. Par récurrence:
		      \begin{itemize}
			      \item Pour $n = 0$, on a $(0,a) \in G$ car $Cl(G)$.
			      \item Pour $n = S(x)$, on a $(n, h(y)) \in G$, pour $(n,y) \in G$.
		      \end{itemize}
		\item Tout élément $n \in \mathbb{N}$ possède une unique image par la relation de graphe $G$. Par récurrence:
		      \begin{itemize}
			      \item Pour $n = 0$ s'il existe $b \neq a$ tel que $(0,b) \in G$. On a que $G' = G \setminus \{(0,b)\}$ vérifie $Cl(G')$, donc
			            $G' \subset G$ ce qui est absurde.
			      \item Si $n = S(m)$. Par hypothèse de récurrence $n$ a une seule image par $G$, qu'on note $X$.On sait aussi que $S(n)$ a pour image $h(x)$.
			            Supposons que $S(n)$ ait pour image $y \neq h(x)$. On pose $G' = G \setminus \{(n,y)\}$ et on va aboutir à une contradiction en montrant que $G'$ vérifie $Cl(G')$.
			            \begin{itemize}
				            \item On note que $(0,a) \in G$ et comme $0 \neq S(n)$, on a $(0,a) \in G'$.
				            \item Soit $d \in \mathbb{N}$ et $z \in E$ tel que $(d,z) \in G'$ alors $(d,z) \in G$ et donc $(S(d), h(z)) \in G$.
				                  \begin{itemize}
					                  \item Si $d \neq m$ et $S(d) \neq S(n)$ alors $(S(d), h(z)) \in G'$.
					                  \item Si $d = m$, donc $z = x$ et alors $(S(d), h(z)) = (m, h(x)) \in G'$.
					                  \item Si $d \neq n$ alors $(S(d), h(z)) \in G'$.
				                  \end{itemize}
				                  Donc $Cl(G')$ ce qui est absurde.
			            \end{itemize}
		      \end{itemize}
	\end{itemize}
\end{proof}


\begin{exercice}
	Démontrer que chacun des axiomes de Peano sont bien nécessaire pour le théorème de Dedekind.
\end{exercice}


\subsection{Quelques propriétés de $\N$}


\begin{prop}
	$\forall x, y, z \in \N, x + (y + z) = (x + y) + z$
\end{prop}

\begin{proof}
	\begin{itemize}
		\item $z = 0$ : $x + (y + 0) = x + y = (x + y) + 0$
		\item $z = S(z')$ :
		      \begin{eqnarray*}
			      x + (y + S(z')) & = & x + S(y + z') \\
			      & = & S(x + (y + z')) \\
			      & = & S((x + y) + z') \\
			      & = & (x + y) + S(z')
		      \end{eqnarray*}
	\end{itemize}
\end{proof}

\begin{prop} Propriétés de l'addition et de la multiplication
	\begin{itemize}
		\item $ 0 + x = 0, \ \forall x \in \N$
		\item $S(x) + y = S(x + y), \ \forall x, y \in \N$
		\item $x + y = y + x, \ \forall x, y \in \N$
		\item $x \cdot (y + z) = x \cdot y + x \cdot z, \ \forall x, y, z \in \N$
		\item $x \cdot ( y \cdot z) = (x \cdot y) \cdot z, \ \forall x, y, z \in \N$
		\item $ 0 \cdot x = 0, \ \forall x \in \N$
		\item $S(x) \cdot y = x \cdot y + y, \ \forall x, y \in \N$
		\item $x \cdot y = y \cdot x, \ \forall x, y \in \N$
		\item $1 \cdot x =  x \cdot 1 =  x, \ \forall x \in \N$
	\end{itemize}
\end{prop}


\begin{example}
	Montrer que, à partir des axiomes ou les propriétés déjà demontrés, on a:
	\begin{eqnarray*}
		z + x  = z' + x &\implies& z = z' \\
		x + z  = y + z' &\implies& z = z'
	\end{eqnarray*}
\end{example}

\begin{proof}
	Si $z + S(n) = z' + S(n)$ alors $S(z + n) = S(z' + n)$.
	Par injectivité de $S$ on a $z + n = z' + n$ et donc $z = z'$.
\end{proof}


\section{La théorie des ensembles}

On note par $A \subset B$ que $\forall x \in A, x \in B$

\subsection{Axiomes de la théorie des ensembles}

\begin{axiom} [Extensionnalité]
	$$\forall A \ \forall B ( A = B \iff \forall x (x \in A \iff x \in B))$$
\end{axiom}

\begin{axiom} [Compréhension]
	Pour toute propriété $P(x)$, exprimée dans le langage de la théorie des ensembles du premier ordre,
	$$ \forall A \ \exists B \subset A \text{ des éĺéments de } A \text{ qui vérifient } P(x)$$
	Qui est équivalent à:
	$$ \forall A \  \exists B \  \forall x (x \in B \iff x \in A \land P(x))$$
	Cet ensemble est unique par extensionnalité et il est noté $\{x \in A \mid P(x)\}$.
\end{axiom}

\begin{axiom}[Des paires]
	$$ \forall A \  \forall B \  \exists C \  \forall \ x \  (x \in C \iff x = A \lor x = B)$$
	Il est noté $\{A,B\}$.
\end{axiom}


\begin{axiom}[Réunion]
	$$ \forall A \ \exists B \ (\forall x \in B \iff (\exists Y \in A  \ x \in  Y))$$
    Noté $\bigcup A$.
\end{axiom}






\end{document}
