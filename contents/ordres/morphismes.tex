\subsection{Morphismes d'ordres}

\begin{definition}

	Un morphisme d'ordres entre 2 ensembles d'ordre ($A$, $\leqslant_A$), ($B$, $\leqslant_B$) est une application $\varphi : A \to B$
	tel que $\forall x$, $y\in A$, $x \leqslant_A y \iff \varphi(x) \leqslant_B \varphi(y)$

\end{definition}

\begin{definition}

	Un isomorphisme d'ordres est un morphisme d'ordres bijectif.

	Remarque : Un isomorphisme est une bijection croissante dont la réciproque est croissante.

\end{definition}

\begin{definition}

	Une relation d'ordre $\leqslant$ sur $E$ définit un bon ordre sur $E$ si :
	\begin{itemize}
		\item $\leqslant$ est un ordre total
		\item Tous sous-ensemble de $E$ non vide a un plus petit élément, c'est-à-dire pour tout $A \subseteq E$, $A \neq \emptyset$, il existe
		      $a \in A$ tel que $\forall b \in A$, $a \leqslant b$
	\end{itemize}

\end{definition}

\begin{example}
	\begin{itemize}
		\item ($\mathbb{N}$, $\leqslant$): Bon ordre
		\item ($\mathbb{Z}$, $\leqslant$): Mauvais ordre car $]-\infty, 0]$ n'est pas borné à gauche
		\item ($[0, 1]$, $\leqslant$): Mauvais ordre ($]0, 1]$ n'admet pas de minimum)
		\item ($\mathscr{P}(\mathbb{N})$, $\subseteq$): Mauvais ordre : Pas total
		\item ($\emptyset$, $\emptyset$) : Bon ordre
	\end{itemize}
\end{example}

\begin{prop}
	Soit ($A$, $\leqslant_A$) un bon ordre. Soit $B \subseteq A$ et soit $\leqslant_B$ une relation d'ordre large de B, alors ($B$, $\leqslant_B$) est bien ordonnée.
\end{prop}

\begin{proof}

	Soit $C \subseteq B$ non vide, alors $C \subseteq A$ non vide. Donc il existe $c \in C$ l'élément le plus petit de $C$ par rapport à $\leqslant_A$. On
	veut que $c$ soit l'élément le plut petit pour $\leqslant_B$. Soit $d \in C$, donc $c \leqslant_A d$ et puis $c \leqslant_B d$.

\end{proof}

\begin{example}
	\begin{itemize}
		\item ($\left\{\frac{1}{n} | n \in \mathbb{N}^*\right\}$, $\leqslant$) n'est pas un bon ordre
		\item ($\left\{-\frac{1}{n} | n \in \mathbb{N}^*\right\}$, $\leqslant$) est un bon ordre : il existe un isomorphisme ($\left\{-\frac{1}{n} | n \in \mathbb{N}^*\right\}$, $\leqslant$) $\to$ ($\mathbb{N}$, $\leqslant$)
	\end{itemize}
\end{example}

\begin{prop}

	Un ensemble non vide total ordonné ($E$, $\leqslant$) est bien ordonné si et seulement si il vérifie la propriété de récurrence bien fondée pour tout sous ensemble $J \subseteq E$ :
	\begin{equation*}
		\forall y \in E \text{, } \forall z \in E \text{, avec } z < y \text{, alors } (z \in J \implies y \in J) \implies J = E
	\end{equation*}
	\noindent

\end{prop}

\begin{proof}

	On reformule par contraposée la proposition de récurrence bien fondée :
	\begin{equation*}
		\exists x \in E\backslash J \implies \exists y \in E\backslash J \text{ tel que } \forall z \in E \text{, } z < y \implies z \in J \text{ ce qui veut dire que } y = min(E\backslash J)
	\end{equation*}
	\noindent
	C'est la proposition des bons ordres pour $E\backslash J \subseteq E$.

\end{proof}

\begin{example}
	Pour $x$, $y \in \mathbb{Z}$, on écrit $x | y$ si et seulement si $\exists z \in \mathbb{Z}$ tel que $y = x\cdot z$
	\begin{itemize}
		\item ($\mathbb{Z}$, $|$) n'est pas un bon ordre car $-1 | 1 | -1$ et $-1 \neq 1$
		\item ($\mathbb{N}$, $|$) est une relation d'ordre pas totale car $2 \nmid 3 \nmid 2$
		\item ($\mathbb{N}$, $\leqslant$) bon ordre car il n'existe pas de suite infinie décroissante dans $\mathbb{N}$
	\end{itemize}
\end{example}
