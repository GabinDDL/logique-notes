\subsection{Axiomatisation et les axiomes de Peano}

\begin{definition}[Axiomes pour les naturels]
	\begin{itemize}
		\item Succeseur non nul: $\forall x \in \mathbb{N}, S(x) \neq 0$
		\item Injectivité du successeur: $\forall x,y \in \mathbb{N}, S(x) = S(y) \implies x = y$
		\item Récurrence: Pour toute prorpiété $P$ << bien définie >> sur $\mathbb{N}$ alors
		      $$ \left( P(0) \wedge \forall x \in \mathbb{N}, P(x) \implies P(S(x))\right) \implies \forall x \in \mathbb{N}, P(x) $$
	\end{itemize}
\end{definition}

\begin{lemma}[Raissonement par récurrence]
	Tout entier est soit $0$ soit un successeur:
	$$ \forall x \in \mathbb{N}, x = 0 \vee \exists y \in \mathbb{N}, x = S(y) $$
\end{lemma}

\begin{proof}
	Par récurrence, la propriété $P(x) = x = 0 \vee \exists y \in \mathbb{N}, x = S(y)$.
	\begin{itemize}
		\item $P(0)$ est vraie car $0 = 0$.
		\item On suppose $P(x)$ vraie, donc $P(S(x))$.
	\end{itemize}
	Notons que dans cette preuve la propriété de récurrence n'est pas utilisée.
\end{proof}


\begin{definition}[Prédécesseur]
	Pour tout élément non nul, on appelle prédécesseur de $x$ un élément $y$ tel que $S(y) = x$.
\end{definition}

\begin{exercice}
	Montrer que les trois axiomes de Peano sont indépendants, c'est à dire, construire pour chacun des axiomes un
	contre-modèle $\mathscr{N}$, avec un élément distingué $0$ et une fonction $S : \mathscr{N} \to \mathscr{N}$, où l’axiome dont on
	veut montrer qu’il est indépendant n’est pas vérifié, mais les deux autres le sont.
\end{exercice}
